% LAMPIRAN
\section*{Lampiran}
\addcontentsline{toc}{section}{Lampiran}

\subsection*{A. Logbook Mingguan}
\addcontentsline{toc}{subsection}{A. Logbook Mingguan}

\begin{table}[H]
\centering
\caption{Logbook Perkembangan Proyek}
\label{tab:logbook}
\small
\begin{tabular}{p{2.5cm}p{5cm}p{6cm}}
\hline
\textbf{Tanggal} & \textbf{Kegiatan} & \textbf{Hasil / Progress} \\ \hline
10/28/2025 & Pembuatan Repositori GitHub Tugas Besar & Repositori GitHub tugas besar berhasil dibuat dengan struktur awal proyek \\ \hline
11/2/2025 & Implementasi Komponen Utama \& Integrasi Aplikasi & Hand tracker dengan MediaPipe, Arpeggiator (kontrol pitch \& volume), Drum Machine (5 pola ritme), Audio Reactive Visualizer, dan integrasi semua komponen di main application. Perbaikan audio system dengan real audio samples, optimisasi code, Custom BPM feature \\ \hline
11/9/2025 & Integrasi PyGame pada proyek dan penambahan fitur & Integrasi Proyek (Visualizer) dari CV2 ke PyGame, Penambahan pattern beat baru dan aset drum terbaru, Perbaikan visualisasi \\ \hline
11/14/2025 & UI baru menggunakan PyQT6, fixing bugs & UI baru menggunakan PyQT6, fixing bugs \\ \hline
11/28/2025 & Perubahan asset dan pattern pada drum & Perubahan asset dan pattern pada drum machine berhasil \\ \hline
11/30/2025 & Remake Pinch BPM Controller & Menerapkan Pinch BPM pada UI baru dan men-sinkronkan perubahan BPM dengan UI \& slider \\ \hline
\end{tabular}
\end{table}

\subsection*{B. Informasi Repository}
\addcontentsline{toc}{subsection}{B. Informasi Repository}

\subsubsection*{Repository GitHub}
Repository proyek ini dapat diakses di:
\begin{itemize}
    \item \textbf{URL}: \url{https://github.com/Alfariz11/Gestune-Musik-dari-gerakan-tangan}
    \item \textbf{Branch Utama}: main
\end{itemize}

\subsection*{C. Teknologi yang Digunakan}
\addcontentsline{toc}{subsection}{C. Teknologi yang Digunakan}

\begin{table}[H]
\centering
\caption{Teknologi dan Library}
\label{tab:technologies}
\begin{tabular}{ll}
\hline
\textbf{Komponen} & \textbf{Teknologi} \\ \hline
Hand Tracking & MediaPipe \\
User Interface & PyQt6 \\
Graphics \& Visualizer & PyGame \\
Audio Processing & PyAudio / Sounddevice \\
Programming Language & Python 3.8+ \\
\hline
\end{tabular}
\end{table}
