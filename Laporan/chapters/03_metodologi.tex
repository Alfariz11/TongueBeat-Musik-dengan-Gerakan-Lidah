% BAB III - CARA PENGGUNAAN
\section{Cara Penggunaan}

\subsection{Menjalankan Aplikasi}

Untuk menjalankan aplikasi Gestune, ikuti langkah-langkah berikut:

\subsubsection{Jalankan Aplikasi Utama}
Jalankan file \texttt{main.py} dengan perintah:

\begin{lstlisting}[language=bash]
python main.py
\end{lstlisting}

\subsubsection{Pengaturan Webcam}
\begin{itemize}
    \item Pastikan webcam Anda aktif dan terhubung dengan baik
    \item Posisikan diri Anda di depan webcam dengan pencahayaan yang cukup
    \item Pastikan kedua tangan terlihat jelas dalam frame kamera
\end{itemize}

\subsection{Kontrol Aplikasi}

\subsubsection{Tangan Kiri - Arpeggiator}
Tangan kiri digunakan untuk mengontrol Arpeggiator dengan dua cara:

\begin{enumerate}
    \item \textbf{Mengatur Pitch Nada}
    \begin{itemize}
        \item Naikkan tangan untuk mendapatkan nada yang lebih tinggi
        \item Turunkan tangan untuk mendapatkan nada yang lebih rendah
        \item Pitch nada akan berubah secara gradual mengikuti posisi vertikal tangan
    \end{itemize}
    
    \item \textbf{Mengatur Volume}
    \begin{itemize}
        \item Lakukan gesture pinch dengan mendekatkan ibu jari dan telunjuk
        \item Semakin kecil jarak antara ibu jari dan telunjuk, semakin kecil volume
        \item Semakin besar jarak antara ibu jari dan telunjuk, semakin besar volume
    \end{itemize}
\end{enumerate}

\subsubsection{Tangan Kanan - Drum Machine}
Tangan kanan digunakan untuk mengontrol Drum Machine dengan lima pola ritme berbeda:

\begin{table}[H]
\centering
\caption{Pola Drum Machine}
\label{tab:drum-patterns}
\begin{tabular}{lll}
\hline
\textbf{Gesture} & \textbf{Pattern} & \textbf{Deskripsi} \\ \hline
Tidak ada jari / Telunjuk & Pattern 1 & Basic 4/4 \\
Telunjuk + Tengah & Pattern 2 & With clap \\
Telunjuk + Tengah + Manis & Pattern 3 & Syncopated \\
Semua kecuali ibu jari & Pattern 4 & Break beat \\
Semua jari & Pattern 5 & Minimal \\
\hline
\end{tabular}
\end{table}

\subsubsection{Kontrol Keyboard}
Selain kontrol dengan tangan, terdapat juga kontrol keyboard:

\begin{itemize}
    \item \textbf{Q atau ESC}: Keluar dari aplikasi
\end{itemize}

\subsection{Tips Penggunaan}

Untuk mendapatkan pengalaman terbaik saat menggunakan aplikasi Gestune, perhatikan tips berikut:

\begin{enumerate}
    \item \textbf{Pencahayaan}: Pastikan ruangan memiliki pencahayaan yang cukup agar hand tracking dapat mendeteksi tangan dengan optimal
    
    \item \textbf{Jarak dengan Webcam}: Jaga jarak yang nyaman dengan webcam (sekitar 50-100 cm)
    
    \item \textbf{Posisi Tangan}: Pastikan kedua tangan terlihat jelas dalam frame kamera dan tidak tertutup oleh objek lain
    
    \item \textbf{Gerakan}: Lakukan gerakan dengan smooth dan tidak terlalu cepat agar sistem dapat mendeteksi dengan akurat
    
    \item \textbf{Visualizer}: Perhatikan Audio Reactive Visualizer yang menampilkan efek partikel sesuai intensitas dan ritme musik yang sedang dimainkan
\end{enumerate}

\subsection{Troubleshooting}

Jika mengalami masalah saat menggunakan aplikasi, coba solusi berikut:

\begin{itemize}
    \item \textbf{Tangan tidak terdeteksi}: Perbaiki pencahayaan atau sesuaikan jarak dengan webcam
    \item \textbf{Audio tidak keluar}: Periksa pengaturan audio sistem Anda atau jalankan \texttt{test\_drums.py} untuk verifikasi
    \item \textbf{Aplikasi lambat}: Tutup aplikasi lain yang berjalan untuk mengalokasikan lebih banyak resource
\end{itemize}
